\chapter{Electrical Systems}

A \alert{resistor} is an electrical component that resists current. The units of
resistance are ohms $(\Omega)$. An important fact involving resistors is
\alert{Ohm's Law}.

\begin{theorem}[Ohm's Law]
    The voltage drop across a resistor is equal to the product of the current passing
    through the resistor and the resistance of the resistor.
    $$V_R = iR.$$
\end{theorem}

A \alert{capacitor} is an electrical component that stores charge, usually in the form
of two parallel plates. A capacitor has a capacitance which is measure in the units of
Farads $(F)$. The current passing through the capacitor is
$$i = C \frac{\dif V_c}{\dif t} = C\dot{V_c}.$$
An \alert{inductor} is an electrical component that resists a change in current.
The inductance of an inductor is measured in Henry's. The voltage drop
across an inductor is
$$V_L = L \frac{\dif i}{\dif t}.$$
Two important laws play a key role in electrical systems: \alert{kirchoff's voltage
law} and \alert{kirchoff's current law}.
\begin{theorem}[Kirchoff's Voltage Law]
    The sum of all voltage drops in a closed loop is equal to zero.
    $$\sum_k V_k = 0.$$
\end{theorem}

\begin{theorem}
    The 
\end{theorem}

