\chapter{Ordinary Differential Equations}
\section{Separable Differential Equations}

\begin{example}
    Find $y$ for the following differential equation:
    $$\frac{dy}{dx} = x.$$
\end{example}
\begin{soln}
    We see that $dy = x ~ dx$. Taking the integral of both sides, we obtain
    $$y = \frac{1}{2} x^2 + C.$$
\end{soln}

\begin{example}
    Find $y$ for the following differential equation:
    $$\frac{dy}{dx} = \frac{1}{x}.$$
\end{example}
\begin{soln}
    We see that $dy = \frac{1}{x} ~ dx$. Taking the integral of both sides,
    we obtain $$y = \ln|x| + C.$$
\end{soln}
\begin{example}
    Find $y$ for the following differential equation:
    $$\frac{dy}{dx} = \cos x.$$
\end{example}
\begin{soln}
    We see that $dy = \cos x ~ dx$. Taking the integral of both sides, we obtain
    $$\boxed{y = \sin x + C}.$$
\end{soln}
\begin{example}
    Find $y$ for the following differential equation:
    $$\frac{dy}{dx} = e^x.$$
\end{example}
\begin{soln}
    We see that $dy = e^x ~ dx$. Taking the integral of borth sides, we obtain
    $$\boxed{y =  e^x + C}.$$
\end{soln}
\begin{example}
    Find $y$ for the following differential equation: 
    $$\frac{dy}{dx} = xy.$$
\end{example}
\begin{soln}
    We see that $\frac{1}{y} ~ dy  = x ~ dx$. Taking the integral of
    both sides, we get
    $$\ln |y| = \frac{1}{2}x^2 + C \implies y = e^{\frac{1}{2}x^2+C}.$$
\end{soln}
\begin{example}
    Find $y$ for the following differential equation:
    $$\frac{dy}{dx} = x^2y-2xy.$$
\end{example}
\begin{soln}
    We see that $dy = xy(x-2) ~ dx \implies \frac{1}{y} ~dy= x^2-2x ~ dx$.
    Taking the integral of both sides we get that
    $$\ln |y| = \frac{1}{3}x^3-x^2 + C \implies y = e^{\frac{1}{3}x^3-x^2 + C}.$$
\end{soln}
\begin{example}
    Find $y$ for the following differential equation:
    $$(x^2+x)~\frac{dy}{dx} = y-1.$$
\end{example}
\begin{soln}
    We see that $\frac{1}{y-1} ~ dy = \frac{1}{x^2+x} ~ dx$. Taking the
    integral of both sides, we find that
    $$\ln|y-1| = \ln|x|-\ln|x+1| +C\implies y = C\left|\frac{x}{x+1}\right|+1.$$
\end{soln}
\begin{example}
    Find $y$ for the following differential equation:
    $$\frac{dy}{dx} = x^2y-2xy,$$
    where $y(0) = e$.
\end{example}
\begin{soln}
    By a previous example we have $$y = C\left(e^{\frac{1}{3}x^3-x^2}\right).$$
    By the initial value, we solve for $C$ to get $C = e$. Therefore,
    $$\boxed{y = e^{\frac{1}{3}x^3-x^2+1}}.$$
\end{soln}

\section{ODEs and PDEs}
\begin{example}[ODE]
    $$\frac{d^2f}{dx^2} + 3 \frac{df}{dx} + 5 = 0  \implies y''+3y'+5=0.$$
\end{example}
The general from of a linear ODE is
$$a_ny^{(n)} +a_{n-1}y^{(n-1)}+\cdots+a_1y'+a_0y = f(x),$$
where $a_i \in \RR$ such that $i \in \{0, 1, \dots, n\}$.
\begin{example}[PDE]
    $$\frac{\partial^2f}{\partial x^2} + \frac{\partial^2 f}{\partial y^2} = 0.$$
\end{example}
The order of a differential equation is the highest derivative in the equation.

For first order ODEs, we try separation of variables first.

\begin{definition}
    A \alert{homogeneous} ordinary differential equation is one with $f(x) = 0$ 
    with constant coefficients.
\end{definition}
\begin{example}[Homogeneous ODE]
    $$y''+7y'+12y=0.$$
\end{example}
Functions of the form $y=e^{mx}$ are an "educated guess" for solving ODEs.
\begin{soln}
    Let $y=e^{mx}$. This implies $y'=me^{mx}$ and $y'' = m^2e^{mx}$.
    Plugging in, we get
    \begin{align*}
        m^2e^{mx}+7me^{mx}+12e^{mx}&=0 \\
        e^{mx}\left(m^2+7m+12\right)&=0 \\
        e^{mx}(m+4)(m+3) &= 0.
    \end{align*}
    Thus we have $y_h = C_1e^{-4x}+C_2e^{-3x}$.
\end{soln}
\begin{example}[Homogeneous ODE with initial values] 
    $$y''+y;-2y=0, y(0) = 4, y'(0)=-5.$$
\end{example}
\begin{soln}
    Let $y=e^{mx}$. This means $y'=me^{mx}$ and $y''=m^2e^{mx}$. Plugging in,
    we get
    \begin{align*}
        m^2e^{mx}+me^{mx}-2e^{mx} &= 0\\
        e^{mx} \left(m^2+m-2\right) &= 0 \\
        e^{mx}(m-1)(m+2) &=0.
    \end{align*}
    So, $m=-2,1$. Therefore $y_h = C_1e^{-2x} + C_2e^{x}$. Then, we get the 
    following system from the initial values:
    \begin{align*}
         C_1+C_2 &= 4 \\
         -2C_1+C_2&=-5.
    \end{align*}
    Solving, we get $C_1 = 3$ and $C_2 = 1$. So, $\boxed{y_h = 3e^{-2x}+e^{x}}$.
\end{soln}
\begin{example}[Homogeneous ODE with repeated roots]
    $$y''+4y'+4y=0.$$
\end{example}
\begin{soln}
    Let $y=e^{mx}$. This means $y'=me^{mx}$ and $y''=m^2e^{mx}$. PLugging in, 
    we get
    \begin{align*}
        m^2e^{mx}+4me^{mx}+4e^{mx} &= 0 \\
        e^{mx}(m+2)^2 &= 0.
    \end{align*}
    Thus, $\boxed{y_1 = C_1e^{-2x}}$. We now proceed to use the  \alert{reduction of order} 
    method.
    So, $y_2 = f(x)\cdot y_1$. First, we have 
    \begin{align*}
        y_2' &= f'(x)y_1+f(x)y_1' = f'(x)e^{-2x}-2f(x)e^{-2x}\\
        y_2'' &= f''(x)y_1+2f'(x)y_1'+f(x)y_1'' = f''(x)e^{-2x}-4f'(x)e^{-2x}+4f(x)e^{-2x}.
    \end{align*}
    Plugging this into the original differential equation, we get
    \begin{align*}
        f''(x)e^{-2x}-4f'(x)e^{-2x}+4f(x)e^{-2x} + 4\left(f'(x)e^{-2x}-2f(x)e^{-2x} \right)
        + 4f(x)e^{-2x} &= 0.
    \end{align*}
    Simplifying yields $f''(x) = 0$. Now, let $g(x) = f'(x)$. This means
    $g'(x) = f''(x)$. So, $g'(x) = 0$. This is equivalent to $\frac{dg}{dx} = 0$.
    Using separation of variables, we get $g(x) = C$. So, $f'(x) = C$. This is
    the same as $\frac{df}{dx} = C \implies f(x) = C_1x+C_2$. Thus,
    $$\boxed{y_2 = \left(C_1x+C_2\right)e^{-2x}}.$$
\end{soln}

\section{Problems}
\begin{problem}
    Find $y$ for the following differential equation and initial values
    $$y''+4y'+5y = 0; ~ y(0) = 1; ~ y'(0) = -4.$$
\end{problem}
\begin{problem}
    Find $y$ for the following differential equation and initial values
    $$y''- 4y'+4y = 0; ~ y(0) = 1; ~ y'(0) = 0.$$
\end{problem}
\begin{problem}
    Find $y$ for the following differential equation and initial values
    $$y'''-7y'+6y = 0; ~ y(0) = 0; ~ y'(0) = 0; ~ y''(0) = 1.$$
\end{problem}

