\chapter{Laplace Transforms}

The \alert{laplace transform} of a function, $f(t)$, is 
$$F(s) = \int_0^{\infty} f(t)e^{-st} \dif t,$$ where $s = a + bi$ for $a > 0$.

\section{Well Known Laplace Transforms}

$\mathcal{L}\{t^nf(t)\} = \frac{\dif^{(n)}}{\dif s ^ {(n)}}\mathcal{L}\{f(t)\}(-1)^n$

\section{Ordinary Differential Equations}

\begin{example}
    Solve for $\omega(t)$: 
    $$\dot{\omega}+2\omega = 6,$$
    given $\omega(0) = 0$.
\end{example}
\begin{soln}
    Taking the laplacian of both sides, we obtain
    $$\mathcal{L} \{ \dot{\omega}+2\omega \} = \mathcal{L} \{6\}
    \implies -\omega(0) + s\Omega(s)+2\Omega(s) = \frac{6}{s}.$$
    Proceeding to solve for $\Omega(s)$ results in
    $$\Omega(s) = \frac{6}{s(s+2)} = \frac{3}{s} - \frac{3}{s+2}.$$
    So, taking the inverse laplace transform gives
    $$w(t) = 3-3e^{-2t}.$$
\end{soln}

\begin{example}
    Solve for $x(t)$:
    $$\ddot{x}+3\dot{x}+2x = 1,$$
    given $x(0) = \dot{x}(0) = 0$.
\end{example}
\begin{soln}
    Taking the laplace transform of both sides, we obtain
    $$\mathcal{L}\{\ddot{x}+3\dot{x}+2x\} = \mathcal{L}\{1\}$$
    Therefore, $$-\dot{x}(0)-sx(0)+s^2X(s)+3(-x(0)+sX(s))+2X(s)=\frac{1}{s}.$$
    Solving for $X(s)$ results in 
    $$X(s) = \frac{1}{s(s^2+3s+2)} = \frac{1}{s(s+1)(s+2)} = \frac{1/2}{s}+\frac{-2}{s+1}+\frac{1/2}{s+2}$$
    Taking the inverse laplace transform of both sides gives
    $$x(t) = \frac{1}{2}-e^{-t}+\frac{1}{2}e^{-2t}.$$
\end{soln}

\begin{example}
    Solve for $y(t)$:
    $$y''+4y'-5y=0,$$
    given $y(0)=1$, and $y'(0)=0$.
\end{example}
\begin{soln}
    Taking the laplace transform of both sides we obtain
    $$\left( -y'(0)-sy(0)+s^2Y(s)\right)+4\left( -y(0)+sY(s)\right)-5Y(s)=0.$$
    Solving for $\mathcal{Y}(s)$, we obtain
    $$\mathcal{Y}(s) = \frac{s+4}{s^2+4s-5} = \frac{5/6}{s-1}+\frac{1/6}{s+5}$$
    Taking the inverse laplace transform of both sides, we obtain
    $$y(t) = \frac{5}{6}e^t+\frac{1}{6}e^{-5t}.$$
\end{soln}
\begin{theorem}[Shifting Theorem]
    $\mathcal{L}\{e^{-at}f(t)\} = F(a+s)$
\end{theorem}

\begin{example}
    Solve for $y(t)$:
    $$y''+4y'+4y = 0,$$
    where $y(0) = 1$, and $y'(0) = 0$.
\end{example}
\begin{soln}
    Taking the laplace transfrom of both sides, we obtain
    $$\left(-y'(0)-sy(0)+s^2\mathcal{Y}(s)\right)+4\left(-y(0)+s\mathcal{Y}(s)\right)+4\mathcal{Y}(s).$$
    Solving for $\mathcal{Y}(s)$,
    $$\mathcal{Y}(s) = \frac{s+4}{s^2+4s+4} = \frac{1}{s+2}+\frac{2}{(s+2)^2}.$$
    Taking the inverse laplace transform of both sides, we have
    $$y(t) = e^{-2t}+2te^{-2t}$$
\end{soln}

\begin{example}
    Solve for $y(t)$:
    $$y''+4y=0,$$
    given $y(0)=1$ and $y'(0)=0$.
\end{example}
\begin{soln}
    Taking the laplace transform of both sides, we obtain
    $$\left( -y'(0)-sy(0)+s^2\mathcal{Y}(s)\right)+4\mathcal{Y}(s) = 0.$$
    Solving for $\mathcal{Y}(s)$,
    $$\mathcal{Y}(s) = \frac{s}{s^2+4}$$
    Therefore, taking the inverse laplace transform of both sides results in
    $$y(t) = \cos 2t.$$
\end{soln}

\begin{example}
    Solve for $y(t)$:
    $$y''+8y'+25y=0,$$
    given $y(0)=1$ and $y'(0)=0$.
\end{example}
\begin{soln}
    Taking the laplace transform of both sides, we obtain
    $$\left(-y'(0)-sy(0)+s^2\mathcal{Y}(s)\right)+8\left( -y(0)+\mathcal{Y}(s)\right)+25\mathcal{Y}(s)=0.$$
    Solving for $\mathcal{Y}(s)$,
    $$\mathcal{Y}(s) = \frac{s+8}{s^2+8s+25} = \frac{s+4}{(s+4)^2+9}+\frac{4}{3}\left(\frac{3}{(s+4)^2+9}\right).$$
    Thus, by the \alert{shifting theorem} and taking the inverse laplace transform of
    both sides results in
    $$y(t) = e^{-4t}\cos(3t) + \frac{4}{3}e^{-4t}\sin(3t).$$
\end{soln}

\begin{example}
    Solve for $y(t)$:
    $$y''-6y'+15y = 2\sin(3t),$$
    given $y(0)=-1$ and $y'(0)=-4$.
\end{example}
\begin{soln}
    Taking the laplace transform of both sides results in
    $$\left( -y'(0)-sy(0)+s^2\mathcal{Y}(s)\right)-6\left( -y(0)+s\mathcal{Y}(s)\right)+15\mathcal{Y}(s) = \frac{2}{s^2+9}.$$
    Solving for $\mathcal{Y}(s)$,
    \begin{align*}
        \mathcal{Y}(s) &= \frac{6}{(s^2+9)(s^2-6s+15)}-\frac{s-2}{s^2-6s+15} \\
                       &= \frac{-s^3+2s^2-9s+24}{(s^2+9)(s^2-6s+15)} \\
                       &= \frac{s+1}{10(s^2+9)}-\frac{11s-25}{10(s^2-6s+15)} \\
                       &= \frac{1}{10}\left(\frac{s+1}{s^2+9}-\frac{11s-25}{(s-3)^2+6}\right) \\
                       &= \frac{1}{10}\left( \frac{s}{s^2+9} + \frac{1}{3}\left(\frac{3}{s^2+9}\right)-\frac{11(s-3)}{(s-3)^2+6}-\frac{8}{\sqrt{6}}\frac{\sqrt{6}}{(s-3)^2+6}\right).
    \end{align*}
    Then, taking the inverse laplace transform, we obtain
    $$y(t) = \frac{1}{10}\left( \cos(3t)+\frac{1}{3}\sin(3t)-11e^{3t}\cos(\sqrt{6}t)-\frac{8}{\sqrt{6}}e^{3t}\sin(\sqrt{6}t)\right).$$
\end{soln}

\section{Systems of Differential Equations}
Laplace transforms can also be used to solve systems of ordinary differential
equations.

\begin{example}
    Solve for $y_1(t)$ and $y_2(t)$:
    \begin{align*}
        y_1' &= -3y_1-10y_2 \\
        y_2' &= y_1+4y_2.
    \end{align*}
    given $y(0) = \begin{pmatrix} 1 \\ 0 \end{pmatrix}$.
\end{example}

\begin{soln}
    Taking the laplace transform of both equations, we find that
    \begin{align*}
        -y_1(0) + s\mathcal{Y}_1(s) &= -3\mathcal{Y}_1(s)-10\mathcal{Y}_2(s) \\
        -y_2(0) + s\mathcal{Y}_2(s) &= \mathcal{Y}_1(s)+4\mathcal{Y}_2(s)
    \end{align*}
    Plugging in the initial values and solving for $\mathcal{Y}_1(s)$ and
    $\mathcal{Y}_2(s)$ gives
    \begin{align*}
        \mathcal{Y}_1(s) &= \frac{s-4}{s^2-s-2} = \frac{5/3}{s+1}-\frac{2/3}{s-2}\\
        \mathcal{Y}_2(s) &= \frac{1}{s^2-s-2} = \frac{1/3}{s-2}-\frac{1/3}{s+1}.
    \end{align*}
    Taking the inverse laplace transfrom of both we obtain the solution to be
    \begin{align*}
        y_1(t) &= \frac{5}{3}e^{-t}-\frac{2}{3}e^{2t} \\
        y_2(t) &= -\frac{1}{3}e^{-t} + \frac{1}{3}e^{2t}.
    \end{align*}
\end{soln}

\begin{example}
    Solve for $y_1(t)$ and $y_2(t)$:
    \begin{align*}
        y_1' &= y_2 \\
        y_2' &= -4y_1,
    \end{align*}
    given $y(0) = \begin{pmatrix} 0 \\ 1 \end{pmatrix}$.
\end{example}

\begin{soln}
    Taking the laplace transform of both equations, we find that
    \begin{align*}
        -y_1(0)+s\mathcal{Y}_1(s) &= \mathcal{Y}_2(s) \\
        -y_2(0)+s\mathcal{Y}_2(s) &= -4\mathcal{Y}_1(s).
    \end{align*}
    Plugging in the initial values ans solving for $\mathcal{Y}_1(s)$ and
    $\mathcal{Y}_2(s)$ gives
    \begin{align*}
        \mathcal{Y}_1(s) &= \frac{1}{s^2+4} = \frac{1}{2}\frac{2}{s^2+4}\\
        \mathcal{Y}_2(s) &= \frac{s}{s^2+4}.
    \end{align*}
    Taking the inverse laplace transform of both we obtain the solution to be
    \begin{align*}
        y_1(t) &= \frac{1}{2}\sin(2t) \\
        y_2(t) &= \cos(2t).
    \end{align*}
\end{soln}

\begin{example}
    Solve for $y_1(t)$ and $y_2(t)$:
    \begin{align*}
        y_1' &= 8y_1 + y_2 \\
        y_2' &= -4y_1 + 4y_2,
    \end{align*}
    given $y(0) = \begin{pmatrix} 1 \\ 0 \end{pmatrix}$
\end{example}
\begin{soln}
    Taking the laplace transfrom of both equations, we find that 
    \begin{align*}
        -y_1(0)+s\mathcal{Y}_1(s) &= 8\mathcal{Y}_1(s) + \mathcal{Y}_2(s) \\
        -y_2(0)+s\mathcal{Y}_2(s) &= -4\mathcal{Y}_1(s)+4\mathcal{Y}_2(s).
    \end{align*}
    Plugging in the initial values and solving for $\mathcal{Y}_1(s)$ and
    $\mathcal{Y}_2(s)$ gives
    \begin{align*}
        \mathcal{Y}_1(s) &= \\
        \mathcal{Y}_2(s) &=
    \end{align*}
\end{soln}

\begin{example}
    Solve for $y(t)$:
    $$ty''-(t+1)y'+ y = 0,$$
    given $y'(0)=0$.
\end{example}

\begin{soln}
    We first notice that
    \begin{align*}
        \mathcal{L}\{ty''\} &= -\frac{\dif}{\dif s} \left(-y'(0)-sy(0)+s^2\mathcal{Y}(s)\right) \\
                            &= -2s\mathcal{Y}(s)-s^2 \frac{\dif \mathcal{Y}(s)}{\dif s} \\
        \mathcal{L}\{ty'\} &= -\frac{\dif }{\dif s}\left(-y(0)+s\mathcal{Y}(s)\right) \\
                           &= -\mathcal{Y}(s)-s \frac{\dif \mathcal{Y}(s)}{\dif s}.
    \end{align*}
    Thus, taking the laplace transform of both sides and rearranging gives
    \begin{align*}
        (s-s^2)\mathcal{Y}'(s)+(2-3s)\mathcal{Y}(s) = 0 &\implies (s-s^2)\mathcal{Y}'(s) = (3s-2)\mathcal{Y}(s) \\
                                                        &\implies \frac{\mathcal{Y}'(s)}{\mathcal{Y}(s)} = \frac{2-3s}{s^2-s} \\
                                                        &\implies \ln \mathcal{Y}(s) = \int -\frac{2}{s}-\frac{1}{s-1} \\
                                                        &\implies \ln \mathcal{Y}(s) = -2\ln | s | - \ln | s - 1 | \\
                                                        &\implies \mathcal{Y}(s)  = \frac{1}{s^2(s-1)} \\
                                                        &\implies \mathcal{Y}(s) = -\frac{1}{s}-\frac{1}{s^2}+\frac{1}{s-1} \\
                                                        &\implies y(t) = e^{t}-t-1.
    \end{align*}
\end{soln}

\begin{example}
    Solve for $y(t)$:
    $$ty''+2y'+4ty=0,$$
    given $y(0) = 1$.
\end{example}
\begin{soln}
    We first notice that
    \begin{align*}
        \mathcal{L}\{ty''\} &= -\frac{\dif}{\dif s} \left(-y'(0)-sy(0)+s^2\mathcal{Y}(s)\right) \\
                            &= -s\mathcal{Y}(s)-s^2 \frac{\dif \mathcal{Y}(s)}{\dif s} \\
        \mathcal{Y}\{ty\} &= 
    \end{align*}
\end{soln}
